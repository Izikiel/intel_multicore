\section{Introducción}
    El objetivo de este trabajo práctico fue inicialmente experimentar con la arquitectura intel, realizando un microkernel de 64 bits inicializando multinúcleo. Más tarde en el desarrollo del proyecto se decidió extender el alcance del mismo, realizando algunos experimentos para analizar las posibles mejoras de rendimiento de algoritmos que se pueden paralelizar en varios núcleos, asimismo se realizaron varios enfoques diferentes en la sincronización entre núcleos: espera activa haciendo polling a memoria versus sincronización con interrupciones inter-núcleo que evitan los accesos al bus de memoria. La ganancia que esperamos obtener, es una reducción considerable en el overhead que genera el manejo de varios hilos sobre sistemas operativos utilizando librerías como por ejemplo pthreads, y de esta forma poder determinar si podría aprovecharse de mejor manera el hardware disponible para resolver problemas de mayor tamaño que el que nos permiten las librerías actuales para multihilo.
    \\

    El informe estará dividido en secciones, cada una describiendo una parte del trabajo.
