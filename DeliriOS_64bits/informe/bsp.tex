\subsection{Inicialización del Bootstrap processor: De estado de especificación multiboot a modo legacy x64}
    \subsubsection{Modo Legacy x64: GDT, Paginación de los primeros 4gb}
    En este punto del booteo estamos en modo protegido, pero según la especificación multiboot debemos utilizar nuestra propia GDT, y lo hacemos, asignamos una GDT con 
    3 entradas todas de nivel 0, una comun a 32 y 64 bits de datos y 2 de código, esto es necesario para hacer los saltos entre modo real y modo protegido y modo protegido-compatibilidad x64 a modo largo x64.

    TODO: IMAGEN DE LA GDT

    Se utilizó un modelo de paginación en identity mapping para los primeros 64 GB de memoria. El modo de paginación elegido fue IA32e en 3 niveles con páginas de 2 megas, notar que el mapeo tuvo que sea realizado en 2 etapas, en la primera se mapearon los primeros 4gb pues desde modo protegido puedo direccionar limitado hasta 4gb y es necesario activar paginación para pasar a modo largo x64, luego ya en modo largo, completamos el esquema de paginacion a 64gb

    TODO: IMAGEN DE LOS NIVELES DE PAGINACION CON SUS FLECHITAS MOSTRANDO EN ROJO QUE POR ENCIMA DE 4GB NO SE PUEDE MAPEAR

    TODO: TABLAS CON DIRECCIONES DE LAS ESTRUCTURAS Y CANTIDAD DE ENTRADAS DE CADA UNA. DATOS ACA ABAJO

    *Paginacion IA32e PML4 -> 0x140000..0x140ffff => 512 entries * 8 bytes(64bits) cada una solo se instancio la primera entrada de PML4

	PDPT -> 0x141000..0x141ffff => 512 entries * 8 bytes(64bits) cada una se instanciaron las primeras 64 entradas nomas, para mapear 64gb

	PDT -> 0x142000..0x181FFF => 32768 entries * 8 bytes(64bits) cada una se instanciaron las 32768 entries para mapear 64 gb con 32768 paginas de 2 megas

	Luego de establecer estas estructuras, realizamos una comprobación de disponibilidad de modo x64, y encendemos los bits del procesador para habilitar dicho modo.

    \subsection{Inicialización del Bootstrap processor: Pasaje a modo largo x64 nativo}

    Para pasar de modo compatibilidad a modo nativo de 64 bits, es necesario realizar un salto largo en la ejecucion a un selector de la GDT de código de 64 bits.\\
    Luego de realizar el salto al segmento de código de x64 de la GDT establecemos un contexto seguro con los registros en cero, seteamos los selectores correspondientes de la GDT y establecemos los punteros a una pila asignada al BSP.

    \subsubsection{Modo Largo x64: Extensión de paginación a 64 gb}

    En este punto ya podemos direccionar arriba de los 4gb, entonces completamos las entradas en las estructuras de paginación para completar el mapeo hasta 64gb.

    TODO: IMAGEN DEL MAPEO COMPLETO EN 3 NIVELES

    \subsubsection{Modo Largo x64: Inicialización del PIC - Captura de excepciones e interrupciones}
	
    Enviamos señales al PIC para programarlo de forma que atienda las interrupciones enmascarables y asignamos una IDT que captura todas las excepciones e interrupciones y de ser necesario, realiza las acciones correspondientes con su ISR asociada. Particularmente las excepciones son capturadas y mostradas en pantalla y se utiliza la interrupcion de reloj para sincronizacion y esperas, las demás interrupciones son ignoradas.

    TODO: IMAGEN DE LA IDT Y FLECHITAS A LAS ISR

    \subsubsection{Modo Largo x64: Mapa de memoria del kernel}

    TODO: IMAGEN DEL MAPA DE MEMORIA DEL KERNEL.
