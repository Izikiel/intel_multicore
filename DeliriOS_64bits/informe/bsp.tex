\subsection{Inicialización del Bootstrap processor: Pasaje entre modo protegido a modo legacy x64}
    \subsubsection{Modo Legacy x64: GDT, Paginación de los primeros 4gb}
    La especificación multiboot nos asegura que estamos en modo protegido, pero no tenemos la certeza de tener una GDT válida, es por esto que asignamos una GDT con 
    3 entradas todas de nivel 0, una comun a 32 y 64 bits de datos y 2 de código, esta diferenciación de descriptores de código es necesaria para realizar los jump far para pasar de modo real hacia modo protegido y de modo protegido-compatibilidad x64 hacia modo largo x64.

    TODO: IMAGEN DE LA GDT

    Se utilizó un modelo de paginación en identity mapping donde se cubren los primeros 64 GB de memoria. El modo de paginación elegido fue IA32e en 3 niveles con páginas de 2 megas, es importante remarcar que como para pasar a modo largo de 64 bits es obligatorio tener paginación activa, el mapeo de la memoria virtual fue realizado en 2 etapas, en la primera se mapearon unicamente los primeros 4gb pues desde modo protegido puedo direccionar como máximo hasta 4gb y luego desde modo largo, se completa el esquema de paginacion a 64gb agregando las entradas necesarias a las estructuras.
    \\

    Esquema de paginación IA32-e:

    \begin{itemize}

        \item \textbf{PML4: } 512 entries disponibles de 8 bytes de ancho cada una. Solo fue necesario instanciar la primera entrada de la tabla.

        \item \textbf{PDPT: } 512 entries disponibles de 8 bytes de ancho cada una. 
        Solo fue necesario instanciar las primeras 64 entradas de esta tabla.

        \item \textbf{PDT: } 32768 entries disponibles de 8 bytes de ancho cada una.
        Se instancian en modo protegido 2048 entradas para cubrir los primeros 4gb y luego desde modo largo se completan las 30720 entradas restantes completando 64 gb.
    \end{itemize}

    TODO: IMAGEN DE LOS NIVELES DE PAGINACION CON SUS FLECHITAS MOSTRANDO EN ROJO QUE POR ENCIMA DE 4GB NO SE PUEDE MAPEAR

    TODO: TABLAS CON DIRECCIONES DE LAS ESTRUCTURAS Y CANTIDAD DE ENTRADAS DE CADA UNA.

	Luego de establecer estas estructuras, realizamos una comprobación de disponibilidad de modo x64, y encendemos los bits del procesador para habilitar dicho modo.

    \subsection{Inicialización del Bootstrap processor: Pasaje a modo largo x64 nativo}

    Para pasar de modo compatibilidad a modo nativo de 64 bits, es necesario realizar un salto largo en la ejecucion a un descriptor de la GDT de código de 64 bits.\\
    \\
    Luego de realizar el salto al segmento de código de x64 de la GDT establecemos un contexto seguro con los registros en cero, seteamos los selectores correspondientes de la GDT y establecemos los punteros a una pila asignada al BSP.

    \subsubsection{Modo Largo x64: Extensión de paginación a 64 gb}

    En este punto ya podemos direccionar arriba de los 4gb, entonces completamos las entradas en las estructuras de paginación para completar el mapeo hasta 64gb.

    TODO: IMAGEN DEL MAPEO COMPLETO EN 3 NIVELES

    \subsubsection{Modo Largo x64: Inicialización del PIC - Captura de excepciones e interrupciones}
	
    Enviamos señales al PIC para programarlo de forma que atienda las interrupciones enmascarables y asignamos una IDT que captura todas las excepciones e interrupciones y de ser necesario, realiza las acciones correspondientes con su ISR asociada. Particularmente las excepciones son capturadas y mostradas en pantalla y se utiliza la interrupcion de reloj para sincronizacion y esperas, las demás interrupciones son ignoradas.

    TODO: IMAGEN DE LA IDT Y FLECHITAS A LAS ISR

    \subsubsection{Modo Largo x64: Mapa de memoria del kernel}

    TODO: IMAGEN DEL MAPA DE MEMORIA DEL KERNEL.
