Como conclusión de este trabajo, pensamos que la performance obtenida al utilizar dual core a un nivel muy bajo para algoritmos paralelizables no es para nada despreciable, dado que obtuvimos mejoras entre 50\% y 100\% dependiendo de la arquitectura del hardware donde se realizaron los experimentos, esto es una mejor performance que utilizando la infraestructura de threads de lenguajes de alto nivel. Restaría investigar mas aplicaciónes en las cuales se pueda aprovechar este modelo ultraeficiente de cómputo y realizar pruebas con sistemas operativos reales que nos permitan verificar que el overhead creado por estos sea suficientemente apreciable como para decidir utilizar nuestra herramienta.