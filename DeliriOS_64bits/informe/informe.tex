\documentclass[a4paper,10pt,twoside]{article}

\usepackage[top=1in, bottom=1in, left=1in, right=1in]{geometry}
\usepackage[utf8]{inputenc}
\usepackage[spanish,es-ucroman,es-noquoting]{babel}
\usepackage{setspace}
\usepackage{fancyhdr}
\usepackage{lastpage}
\usepackage{amsmath}
\usepackage{amsfonts}
\usepackage{verbatim}
\usepackage{listings}
\usepackage{graphicx}
\usepackage{float}
\usepackage{algorithmic}
\usepackage{color}
\usepackage{hyperref}
\usepackage[usenames,dvipsnames]{xcolor}
\definecolor{dkgreen}{rgb}{0,0.6,0}
\definecolor{gray}{rgb}{0.97,0.97,0.97}
\definecolor{mauve}{rgb}{0.58,0,0.82}
\usepackage{tikz}
\usetikzlibrary{calc}
\usetikzlibrary{decorations.pathreplacing}
\usepackage{ragged2e}

\lstset{
    backgroundcolor=\color{lbcolor},
    tabsize=4,
    rulecolor=,
    language=matlab,
        basicstyle=\scriptsize,
        upquote=true,
        aboveskip={1.5\baselineskip},
        columns=fixed,
        showstringspaces=false,
        extendedchars=true,
        breaklines=true,
        prebreak = \raisebox{0ex}[0ex][0ex]{\ensuremath{\hookleftarrow}},
        frame=single,
        showtabs=false,
        showspaces=false,
        showstringspaces=false,
        identifierstyle=\ttfamily,
        keywordstyle=\color[rgb]{0,0,1},
        commentstyle=\color[rgb]{0.133,0.545,0.133},
        stringstyle=\color[rgb]{0.627,0.126,0.941},
}

% Evita que el documento se estire verticalmente para ocupar
% el espacio vacío en cada página.
\raggedbottom


%%%%%%%%%% Configuración de Fancyhdr - Inicio %%%%%%%%%%
\pagestyle{fancy}
\thispagestyle{fancy}
\lhead{Trabajo Práctico Final, Organización del Computador II}
\rhead{DeliriOS}
\renewcommand{\footrulewidth}{0.4pt}
\cfoot{\thepage /\pageref{LastPage}}

\fancypagestyle{caratula} {
   \fancyhf{}
   \cfoot{\thepage /\pageref{LastPage}}
   \renewcommand{\headrulewidth}{0pt}
   \renewcommand{\footrulewidth}{0pt}
}
%%%%%%%%%% Configuración de Fancyhdr - Fin %%%%%%%%%%


%%%%%%%%%% Configuración de Algorithmic - Inicio %%%%%%%%%%
% Entorno propio para customizar la presentación del pseudocódigo
\newenvironment{pseudocodigo}
    {\vspace{0.5em} \begin{algorithmic}}
    {\end{algorithmic} \vspace{0.5em}}

% Alinear comentarios a la derecha
\renewcommand{\algorithmiccomment}[1]{\hfill \{#1\}}
%%%%%%%%%% Configuración de Algorithmic - Fin %%%%%%%%%%


%%%%%%%%%% Macros de tikz - Inicio %%%%%%%%%%
% Uso: \registroCuatro{etiqueta}{x}{y}{a4}{a3}{a2}{a1}
\newcommand{\registroCuatro}[7]{
    \ifthenelse{\equal{#1}{}}{}{
        \draw (#2, {#3 + 0.5}) node[anchor=east]{#1};
    }

    \draw   (#2, #3) rectangle +(4, 1) +(2, 0.5) node{#4}
          ++(4, 0)   rectangle +(4, 1) +(2, 0.5) node{#5}
          ++(4, 0)   rectangle +(4, 1) +(2, 0.5) node{#6}
          ++(4, 0)   rectangle +(4, 1) +(2, 0.5) node{#7};
}

% Uso: \registroOcho{etiqueta}{x}{y}{a8}{a7}{a6}...{a1}
\newcommand{\registroOcho}[9]{
    \def\etiqueta{#1}
    \def\x{#2}
    \def\y{#3}
    \def\aviii{#4}
    \def\avii{#5}
    \def\avi{#6}
    \def\av{#7}
    \def\aiv{#8}
    \def\aiii{#9}
    \registroOchoX
}
\newcommand{\registroOchoX}[2]{ % Auxiliar - no usar directamente
    \def\aii{#1}
    \def\ai{#2}
    \ifthenelse{\equal{\etiqueta}{}}{}{
        \draw (\x, {\y + 0.5}) node[anchor=east]{\etiqueta};
    }
    \filldraw[fill=white]
        (\x, \y) rectangle +(2, 1) +(1, 0.5) node{\aviii}
        ++(2, 0) rectangle +(2, 1) +(1, 0.5) node{\avii}
        ++(2, 0) rectangle +(2, 1) +(1, 0.5) node{\avi}
        ++(2, 0) rectangle +(2, 1) +(1, 0.5) node{\av}
        ++(2, 0) rectangle +(2, 1) +(1, 0.5) node{\aiv}
        ++(2, 0) rectangle +(2, 1) +(1, 0.5) node{\aiii}
        ++(2, 0) rectangle +(2, 1) +(1, 0.5) node{\aii}
        ++(2, 0) rectangle +(2, 1) +(1, 0.5) node{\ai};
}


% Uso: \registroDieciseis{etiqueta}{x}{y}{a16}{a15}{a14}...{a1}
\newcommand{\registroDieciseis}[9]{
    \def\etiqueta{#1}
    \def\x{#2}
    \def\y{#3}
    \def\axvi{#4}
    \def\axv{#5}
    \def\axiv{#6}
    \def\axiii{#7}
    \def\axii{#8}
    \def\axi{#9}
    \registroDieciseisX
}
\newcommand{\registroDieciseisX}[9]{ % Auxiliar - no usar directamente
    \def\ax{#1}
    \def\aix{#2}
    \def\aviii{#3}
    \def\avii{#4}
    \def\avi{#5}
    \def\av{#6}
    \def\aiv{#7}
    \def\aiii{#8}
    \def\aii{#9}
    \registroDieciseisXX
}
\newcommand{\registroDieciseisXX}[1]{ % Auxiliar - no usar directamente
    \def\ai{#1}
    \ifthenelse{\equal{\etiqueta}{}}{}{
        \draw (\x, {\y + 0.5}) node[anchor=east]{\etiqueta};
    }
    \filldraw[fill=white]
        (\x, \y) rectangle +(1, 1) +(0.5, 0.5) node{\axvi}
        ++(1, 0) rectangle +(1, 1) +(0.5, 0.5) node{\axv}
        ++(1, 0) rectangle +(1, 1) +(0.5, 0.5) node{\axiv}
        ++(1, 0) rectangle +(1, 1) +(0.5, 0.5) node{\axiii}
        ++(1, 0) rectangle +(1, 1) +(0.5, 0.5) node{\axii}
        ++(1, 0) rectangle +(1, 1) +(0.5, 0.5) node{\axi}
        ++(1, 0) rectangle +(1, 1) +(0.5, 0.5) node{\ax}
        ++(1, 0) rectangle +(1, 1) +(0.5, 0.5) node{\aix}
        ++(1, 0) rectangle +(1, 1) +(0.5, 0.5) node{\aviii}
        ++(1, 0) rectangle +(1, 1) +(0.5, 0.5) node{\avii}
        ++(1, 0) rectangle +(1, 1) +(0.5, 0.5) node{\avi}
        ++(1, 0) rectangle +(1, 1) +(0.5, 0.5) node{\av}
        ++(1, 0) rectangle +(1, 1) +(0.5, 0.5) node{\aiv}
        ++(1, 0) rectangle +(1, 1) +(0.5, 0.5) node{\aiii}
        ++(1, 0) rectangle +(1, 1) +(0.5, 0.5) node{\aii}
        ++(1, 0) rectangle +(1, 1) +(0.5, 0.5) node{\ai};
}
%%%%%%%%%% Macros de tikz - Fin %%%%%%%%%%


%%%%%%%%%% Macros misceláneos - Inicio %%%%%%%%%%
\newcommand{\xmm}[1]{\texttt{XMM#1}}
\newcommand{\rax}{\texttt{RAX}}
\newcommand{\rbx}{\texttt{RBX}}
\newcommand{\rcx}{\texttt{RCX}}
\newcommand{\rdx}{\texttt{RDX}}
\newcommand{\rbp}{\texttt{RBP}}
\newcommand{\rsp}{\texttt{RSP}}
\newcommand{\reg}[1]{\texttt{R#1}}
\newcommand{\asm}[1]{\texttt{\uppercase{#1}}}
%%%%%%%%%% Macros misceláneos - Fin %%%%%%%%%%


\begin{document}


%%%%%%%%%%%%%%%%%%%%%%%%%%%%%%%%%%%%%%%%%%%%%%%%%%%%%%%%%%%%%%%%%%%%%%%%%%%%%%%
%% Carátula                                                                  %%
%%%%%%%%%%%%%%%%%%%%%%%%%%%%%%%%%%%%%%%%%%%%%%%%%%%%%%%%%%%%%%%%%%%%%%%%%%%%%%%


\thispagestyle{caratula}

\begin{center}

\includegraphics[height=2cm]{images/DC.png}
\hfill
\includegraphics[height=2cm]{images/UBA.jpg}

\vspace{2cm}

Departamento de Computación,\\
Facultad de Ciencias Exactas y Naturales,\\
Universidad de Buenos Aires

\vspace{4cm}

\begin{Huge}
Trabajo Práctico Final: DeliriOS
\end{Huge}

\vspace{0.5cm}

\begin{Large}
Organización del Computador II
\end{Large}

\vspace{1cm}

Marzo 2014

\vspace{4cm}

\begin{tabular}{|c|c|c|}
    \hline
    Apellido y Nombre & LU & E-mail\\
    \hline
    Silvio Vileriño             & 106/12 & svilerino@gmail.com\\
    Ezequiel Gambaccini 		& 715/13 & ezequiel.gambaccini@gmail.com\\
    \hline
\end{tabular}
\vspace{1cm}
\\
\textbf{Repositorio del Proyecto:}\\
\url{https://github.com/Izikiel/intel_multicore}


\end{center}
\newpage


%%%%%%%%%%%%%%%%%%%%%%%%%%%%%%%%%%%%%%%%%%%%%%%%%%%%%%%%%%%%%%%%%%%%%%%%%%%%%%%
%% Índice                                                                    %%
%%%%%%%%%%%%%%%%%%%%%%%%%%%%%%%%%%%%%%%%%%%%%%%%%%%%%%%%%%%%%%%%%%%%%%%%%%%%%%%

\tableofcontents

\newpage

    %%%%%%%%%%%%%%%%%%%%%%%%%%%%%%%%%%%%%%%%%%%%%%%%%%%%%%%%%%%%%%%%%%%%%%%%%%%%%%%
    %% Introduccion                                                              %%
    %%%%%%%%%%%%%%%%%%%%%%%%%%%%%%%%%%%%%%%%%%%%%%%%%%%%%%%%%%%%%%%%%%%%%%%%%%%%%%%
    \section{Introduccion}
    El objetivo de este trabajo práctico fue inicialmente experimentar con la arquitectura intel, realizando un microkernel de 64 bits inicializando multinúcleo. Más tarde en el desarrollo del proyecto se decidió extender el alcance del mismo, realizando algunos experimentos para analizar las posibles mejoras de rendimiento de algoritmos que se pueden paralelizar en varios núcleos, asimismo se realizaron varios enfoques diferentes en la sincronizacion entre núcleos: espera activa haciendo polling a memoria versus sincronizacion con interrupciones inter-núcleo que evitan los accesos al bus de memoria. La ganancia que esperamos obtener, es una reducción considerable en el overhead que genera el manejo de varios hilos sobre sistemas operativos utilizando librerias como por ejemplo pthreads, y de esta forma poder determinar si podría aprovecharse de mejor manera el hardware disponible para resolver problemas de mayor tamaño que el que nos permiten las librerias actuales para multihilo.
    \\

    El informe estará dividido en secciones, cada una describiendo una parte del trabajo.

    \newpage

    %%%%%%%%%%%%%%%%%%%%%%%%%%%%%%%%%%%%%%%%%%%%%%%%%%%%%%%%%%%%%%%%%%%%%%%%%%%%%%%
    %% Desarrollo                                                                %%
    %%%%%%%%%%%%%%%%%%%%%%%%%%%%%%%%%%%%%%%%%%%%%%%%%%%%%%%%%%%%%%%%%%%%%%%%%%%%%%%
    \section{Desarrollo: Inicialización y contexto del sistema}
            \subsection{Estructura de carpetas: Compilación, Linkeo y Scripts}
    \begin{itemize}
	    \item \textbf{ap code: } Contiene el código de los Application processors, tanto de inicialización desde modo protegido como de los algoritmos implementados.
		\item \textbf{ap startup code: } Contiene el código de inicializacion en modo real de los Application processors
		\item \textbf{bsp code: } Contiene el código de inicializacion del kernel, los algoritmos implementados, y el encendido de los Application Processors.
		\item \textbf{common code: } Contiene el código comun al Bootstrap Processor y los Application Processors.
		\item \textbf{grub init: } Contiene scripts y archivos de configuracion de grub, en la subcarpeta src se encuentra el loader que realiza el pasaje entre la máquina en especificación multiboot a modo largo de 64 bits.
		\item \textbf{run.sh: } Script para compilación y distribución del tp.
		\item \textbf{macros: } Esta carpeta contiene macros utilizadas en el código.
		\item \textbf{informe: } Contiene el informe del trabajo práctico en formato Latex y PDF.
		\item \textbf{include: } Contiene las cabeceras de las librerias.
    \end{itemize}

    El tp está compilado en varios archivos, de esta forma cargamos algunas partes como módulos de grub.
    Hay scripts encargados de compilar todo lo necesario, cada módulo tiene su Makefile y el comando make es llamado
    oportunamente por los scripts en caso de ser necesario.
    \\
    \\
    El linkeo está realizado con linking scripts en las carpetas donde sea necesario,
    cada seccion esta alineada a 4k por temas de compatibilidad, asimismo hay símbolos que pueden ser leidos desde el codigo si es necesario saber la ubicación de estas secciones en memoria.
    Los módulos de 32 bits estan compilados en elf32 y los de 64 bits en binario plano de 64 bits, esto es por temas de compatibilidad de grub al cargar modulos.
    \\
    \\
    Para correr el trabajo practico unicamente hace falta tipear $./run.sh -r$ en consola y se abrira en bochs el trabajo practico.

        \newpage
         \subsection{Integración con grub y división en módulos}
	Utilizamos grub para la parte del booteo para lograr un contexto inicial estable y que posibilitara la ejecución del trabajo practico desde un pendrive usb.
	Grub permite iniciar un kernel por medio de una especificación publicada en la web, en la que se detalla un contrato que debe cumplir tanto el kernel a iniciar como grub, entre ellas, un header que debe contener el kernel para ser identificado por grub como un kernel, y grub debe otorgarle control a dicho kernel en un estado determinado, es decir, una gdt, los registros con valores predeterminados, etc.

	Esta revisión de grub no permite cargar kernels compilados en 64 bits de manera sencilla, es por esto que recurrimos a una herramienta que provee grub, que son los módulos, de esta forma podemos cargar por encima del mega distintas partes del sistema y tener claro en que parte de memoria estan cargadas.

	Para solucionar el inconveniente de iniciar un kernel compilado en 64 bits, realizamos un booteo en etapas, cargando módulos y recurriendo a ellos cuando es necesario. En las próximas secciones se explicará esto con más detalle.

	\subsubsection{Booteo e integración con grub: Especificación multiboot, carga de modulos elf32 y binarios x64}
		El kernel que inicia grub debe ser un ejecutable elf32, que además cuente con un header especificado por la convencion de grub. Esto nos provee un punto de inicio del kernel en donde se sabe exactamente el estado de la máquina (\url{http://www.gnu.org/software/grub/manual/multiboot/multiboot.html#Machine-state}).
		En este primer nivel de booteo, se realizan varias comprobaciones de la especificacion multiboot, luego son inicializadas variables globales para comunicacion entre módulos y se copia el codigo inicial de los Application Processors a una posicion de memoria alineada a página de 4K por debajo del primer mega, Además son pasadas como parámetros unas estructuras de datos que contienen informacion del sistema, lo que nos interesa a nosotros en particular es una lista en donde se encuentran los módulos que fueron cargados y las posiciones de memoria en donde se encuentran copiados. Una vez obtenidas las posiciones de memoria de los módulos, realizamos un salto en la ejecución al módulo de inicializacion del Bootstrap Processor, compilado como binario plano de 64 bits.

	\subsubsection{Booteo e integración con grub: Mapa de memoria: Memoria baja y modulos en memoria alta}

		TODO: LISTA DE MODULOS Y QUE PARTE DEL TP ES CADA UNO

		TODO: INSERTAR GRAFICO DE MEMORIA A LO TP3 DE ORGA2.

		TODO: INSERTAR GRAFICO DE TODOS LOS MODULOS, A LO UML, INDICANDO LAS VARIABLES GLOBALES QUE COMPARTEN,
		IE. EL 0xABBAABBA PARA HACER EL JUMP DE LOS AP Y EL JMP AL BINARIO DE 64 BITS DESDE EL LOADER DE GRUB

        \newpage
        \subsection{Inicialización del Bootstrap processor: Pasaje entre modo protegido a modo legacy x64}
    \subsubsection{Modo Legacy x64: GDT, Paginación de los primeros 4gb}
    La especificación multiboot nos asegura que estamos en modo protegido, pero no tenemos la certeza de tener una GDT válida, es por esto que asignamos una GDT con 
    3 entradas todas de nivel 0, una comun a 32 y 64 bits de datos y 2 de código, esta diferenciación de descriptores de código es necesaria para realizar los jump far para pasar de modo real hacia modo protegido y de modo protegido-compatibilidad x64 hacia modo largo x64.

    TODO: IMAGEN DE LA GDT

    Se utilizó un modelo de paginación en identity mapping donde se cubren los primeros 64 GB de memoria. El modo de paginación elegido fue IA32e en 3 niveles con páginas de 2 megas, es importante remarcar que como para pasar a modo largo de 64 bits es obligatorio tener paginación activa, el mapeo de la memoria virtual fue realizado en 2 etapas, en la primera se mapearon unicamente los primeros 4gb pues desde modo protegido puedo direccionar como máximo hasta 4gb y luego desde modo largo, se completa el esquema de paginacion a 64gb agregando las entradas necesarias a las estructuras.
    \\

    Esquema de paginación IA32-e:

    \begin{itemize}

        \item \textbf{PML4: } 512 entries disponibles de 8 bytes de ancho cada una. Solo fue necesario instanciar la primera entrada de la tabla.

        \item \textbf{PDPT: } 512 entries disponibles de 8 bytes de ancho cada una. 
        Solo fue necesario instanciar las primeras 64 entradas de esta tabla.

        \item \textbf{PDT: } 32768 entries disponibles de 8 bytes de ancho cada una.
        Se instancian en modo protegido 2048 entradas para cubrir los primeros 4gb y luego desde modo largo se completan las 30720 entradas restantes completando 64 gb.
    \end{itemize}

    TODO: IMAGEN DE LOS NIVELES DE PAGINACION CON SUS FLECHITAS MOSTRANDO EN ROJO QUE POR ENCIMA DE 4GB NO SE PUEDE MAPEAR

    TODO: TABLAS CON DIRECCIONES DE LAS ESTRUCTURAS Y CANTIDAD DE ENTRADAS DE CADA UNA.

	Luego de establecer estas estructuras, realizamos una comprobación de disponibilidad de modo x64, y encendemos los bits del procesador para habilitar dicho modo.

    \subsection{Inicialización del Bootstrap processor: Pasaje a modo largo x64 nativo}

    Para pasar de modo compatibilidad a modo nativo de 64 bits, es necesario realizar un salto largo en la ejecucion a un descriptor de la GDT de código de 64 bits.\\
    \\
    Luego de realizar el salto al segmento de código de x64 de la GDT establecemos un contexto seguro con los registros en cero, seteamos los selectores correspondientes de la GDT y establecemos los punteros a una pila asignada al BSP.

    \subsubsection{Modo Largo x64: Extensión de paginación a 64 gb}

    En este punto ya podemos direccionar arriba de los 4gb, entonces completamos las entradas en las estructuras de paginación para completar el mapeo hasta 64gb.

    TODO: IMAGEN DEL MAPEO COMPLETO EN 3 NIVELES

    \subsubsection{Modo Largo x64: Inicialización del PIC - Captura de excepciones e interrupciones}
	
    Enviamos señales al PIC para programarlo de forma que atienda las interrupciones enmascarables y asignamos una IDT que captura todas las excepciones e interrupciones y de ser necesario, realiza las acciones correspondientes con su ISR asociada. Particularmente las excepciones son capturadas y mostradas en pantalla y se utiliza la interrupcion de reloj para sincronizacion y esperas, las demás interrupciones son ignoradas.

    TODO: IMAGEN DE LA IDT Y FLECHITAS A LAS ISR

    \subsubsection{Modo Largo x64: Mapa de memoria del kernel}

    TODO: IMAGEN DEL MAPA DE MEMORIA DEL KERNEL.

        \newpage
        \subsection{Multicore: encendido de los AP's}

Una vez inicializado el bsp, se procede a inicializar el resto de los procesadores del sistema.\\
Para lograr esto, primero es necesario que encontrar una estructura de datos llamada MP Floating Pointer Structure, que contiene la información sobre: los demas procesadores; apic (advanced programable interrupt controller); y el I/O apic (análogo al apic, pero se encarga además de rutear interrupciones de input/output a los lapic's).

\subsubsection{Búsqueda de la estructura MP Floating Pointer}
Esta estructura puede estar en diferentes lugares de memoria, inicialmente se debe realizar la búsqueda dentro del primer kilobyte de la ebda (extended bios data area), de no encontrarse allí, se procede a buscar entre los 639K-640K de memoria.
Para encontrar la tabla se busca dentro de esas areas de memoria la firma de la MP Floating Pointer Structure, la cual es "\_MP\_".

\subsubsection{Comprobación de checksum de la estructura}
Una vez encontrada una estructura con esa firma, es necesario realizar una comprobacion de checksum para verificar que la misma sea válida.
De no ser válida la estructura encontrada, se continúa buscando dentro de las areas de memoria previamente mencionadas, al agotarse las áreas mencionadas sin éxito en la buqueda, se concluye que el sistema no soporta multicore.

\subsubsection{Habilitacion de ICMR y Local APIC}
El sistema luego es configurado usando la MP Configuration Table, extraída de la MP Floating Pointer Structure, que contiene en diferentes entradas la información acerca de los diferentes dispositivos utilizados en multicore.
Una vez parseadas estas entradas, si la maquina tiene ICMR, se habilita ese registro, y luego se procede a inicializar el apic local del bsp, seteando el vector de interrupciones espurias (interrupciones por ruido de hardware, etc) y habilitando el bit de enable dentro de los registros del local apic. (Registros del local apic especificados en Table 10-1 Local APIC Register Address Map, manual 3 de intel capitulo 10).

\subsubsection{Inicializacion de los AP's}

Una vez finalizada la inicializacion del local apic, se debe pedir a la BIOS que setee el warm reset vector a la dirección donde esta localizado el inicio de modo real de los aps$(0x2000)$. Luego de este paso, se procede a encender los aps usando la información obtenida por la MP Configuration Table.

El proceso de encendido los aps consiste en preparar 3 estructuras de interrupt command register (registro del apic usado para ipis) para luego enviar las ipis especificas de inicio. El primer icr se setea con el delivery mode de INIT, que sea una interrupción por nivel y con la dirección del ap a iniciar. El segundo icr se setea con el delivery mode de INIT\_DASSERT, que sea una interrupción por flanco, y que sea enviado a todos los procesadores. El tercer icr se setea con el delivery mode STARTUP, la dirección fisica de la página de inicio de ejecución del ap shifteado 12 a derecha, y la dirección del ap.

Luego de tener listos los icr, se procede a mandar las ipis de INIT e INIT\_DASSERT, luego se espera unos 10 milisegundos, verificando previamente que las ipis se hayan enviado correctamente. Luego de la espera, se procede a enviar la ipi de STARTUP, se espera 10 ms, se verifica que se haya enviado correctamente, se la vuelve a enviar, se realiza otra espera de 10 ms, y se vuelve a verificar.

Una vez terminado este proceso, se puede asumir que el ap fue encendido, y se continúa el encendido el resto de los aps encontrados en la MP Configuration Table.\\

\textbf{Nota 1: } El proceso de inicializacion de los aps se encuentra especificado en detalle en el manual de intel volumen 3, capitulo 8, subsección 4, MULTIPLE-PROCESSOR (MP) INITIALIZATION.

\textbf{Nota 2: } Dado que se realizan experimentos con un máximo de 2 cores, se limita el encendido de los Application Processors a uno, que junto con el Bootstrap Processor son 2 núcleos.\\

\textbf{Nota 3: } Los identificadores de nucleo no necesariamente son valores numéricos consecutivos.
        \newpage
        \subsection{Multicore: inicialización de modo real a modo nativo x64 de los AP's}
    Como vimos en la sección anterior, los Application Processors comienzan su ejecución en una posición otra por debajo del primer mega en modo real, nosotros necesitamos hacer saltar la ejecución a una posición conocida por encima del mega que no se solape con estructuras del kernel ni otros módulos, la solución que propusimos es un booteo por etapas.
    
    \subsubsection{Booteo por niveles: Modo real a modo protegido y modo protegido en memoria alta}
    En este primer nivel el núcleo se encuentra en modo real, se inicializa una GDT básica en el mismo código y se salta a modo protegido, esto es necesario para poder direccionar posiciones de memoria por encima del primer mega.\\

    Recordando secciones anteriores, cuando el BSP iniciaba el primer nivel de booteo preparaba el contexto de los APS para inicializar en niveles, en este proceso se inyecta en el código del primer nivel de booteo del AP la posicion de memoria donde esta el segundo nivel de booteo por encima del mega.\\

    Luego resta únicamente realizar el salto al segundo bootloader con un jump para continuar el booteo del AP, de manera similar al BSP pasamos luego a modo nativo de 64 bits.\\
    Notemos que por ejemplo la línea A20 ya esta habilitada, y algunas estructuras del kernel que fueron inicializadas por el BSP son comúnes a todos los núcleos, como por ejemplo la GDT y la estructura jerárquica de paginación,
    que son directamente asignadas a los registros del núcleo correspondiente con punteros a ellas.
    \\
    Para el manejo de interrupciones y excepciones, se inicializa una IDT para los Application Processors, además es necesario habilitar los local-apic de cada AP, en caso contrario, no podríamos utilizar interrupciones inter-procesador(IPIS).
    \\

    Como el número de application processors puede ser variable a priori, cuando un núcleo comienza su ejecución, es obtenido su código de identificación dentro del procesador y luego se obtiene una posición de memoria única para cada núcleo con el fin de poder asignar los punteros de la pila $RSP$ y $RBP$, de ser necesaria la reserva de memoria para alguna otra estructura única por núcleo se puede utilizar este recurso.
        \newpage
    \section{Desarrollo: Algoritmos implementados}
        En esta sección, con el contexto del sistema inicializado completamente, describiremos los algoritmos paralelizados implementados y sus resultados.
        \subsection{Sorting de arreglos}	
    \subsubsection{Conjuntos de numeros pseudoaleatorios utilizados para los experimentos}
    	Para este experimento se realizaron pruebas con diferentes longuitudes de arreglo, estas siempre
    	siendo una potencia de 2.\\
    	Las muestras de datos se generaron utilizando un generador de numeros pseudo-aleatorios
    	congruencial lineal con la misma semilla inicial.
        Asimismo luego de todos los ordenamientos de distinto tamaño de entrada y modo de procesamiento, se realiza una verificación de correctitud del algoritmo de ordenamiento para detectar posibles errores en el procesamiento o sincronización de los núcleos.
    \subsubsection{Implementación con un unico núcleo}
    	En este experimento se realiza un heap-sort sobre todos los arreglos de distinto tamaño.
    \subsubsection{Implementación con dos cores: Paralelización del algoritmo}
    	Se optó en este caso por un algoritmo propio que combinara varios ordenamientos conocidos con el objetivo de hacer mas facil la paralelización del experimento. En particular lo que se realiza es partir el arreglo en 2 mitades, donde cada core realiza heapsort sobre su mitad asignada y luego una intercalación de los resultados con un algoritmo tambien paralelo en el cual un core realiza el merge de los maximos y el otro el merge de los minimos, luego resta copiar los resultados y concluimos.

   	\begin{center}
        \includegraphics[height=9cm]{images/dualcore-sorting.pdf}
    \end{center}
    	

    \subsubsection{Implementación con dos cores: Sincronización con espera activa}
    	Para abordar el problema de la sincronización la primera implementacion fue basada en espera activa, en la que basicamente existen flags en los cuales los distintos núcleos indican su estado actual entre sí actualizando los diferentes flags, igualmente esto no es simétrico, hay 2 roles definidos, maestro y esclavo, el primero da las órdenes y espera la señal de finalización de las operaciones asignadas al núcleo esclavo.
    	El algoritmo funciona de la siguiente manera, siendo el bsp el master y el ap el slave:
    	\begin{enumerate}
    		\item Cuando el master empieza a ejecutar el algoritmo, el slave esta chequeando todo el tiempo una variable para ver si puede iniciar o no el trabajo.
    		
    		\item El master setea el flag de inicio del sort y empieza a ordenar su mitad del arreglo, mientras que el slave sale del loop de verificacion y empieza a ordenar su mitad del arreglo.
    		
    		\item Una vez que el master termina, se queda esperando a que el slave setee el flag de que termino la etapa en la que estaba. El slave cuando termina su etapa setea el flag que avisa que termino y se pone a chequear en un loop un flag para que le autorice a pasar a la siguiente etapa.

    		\item Cuando el master verifica que el slave termino, resetea los flags utilizados hasta ahora, y setea el flag de inicio de la etapa de merge, luego de lo cual se pone a realizar merge de su mitad. El slave, por su parte, una vez que verifica que puede pasar a la siguiente etapa, se pone hacer merge con su mitad, seteando un flag para avisar que termino.

    		\item El master, luego de verificar que el slave termino, setea el flag para iniciar la última etapa del algoritmo, que es la de copiar los resultados de los diferentes merge al arreglo original. Cada core por su lado copia los resultados de sus respectivos merge al array original. El slave cuando termina le avisa al master y se queda loopeando esperando la señal para arrancar el algoritmo de 0.

    		\item Una vez que el master verifico que el slave termino, resetea todos los flags utilizados hasta ahora, y sale de la función.

			\item Cuando se terminan todas las pruebas, se setea una variable global para avisarle al slave que puede salir de la función y pasar a modo de bajo consumo.
    	\end{enumerate}

   	\begin{center}
        \includegraphics[height=5.5cm]{images/sync-sorting-seq.png}
    \end{center}

    \subsubsection{Implementación con dos cores: Sincronización con inter-processor interrupts}
    	En esta implementación, el slave se encuentra en modo de bajo consumo, y lo que hace el master es mandarle interrupciones específicas de acuerdo a la etapa en la que esta del algoritmo. En total son 3 interrupciones diferentes, las cuales hacen que el slave ordene, mergee y copie respectivamente.\\

    	En este caso el protocolo de sincronización para las tres etapas es:
    	\begin{enumerate}
    		\item El slave espera que el master envie una interrupcion para comenzar la etapa correspondiente.
    		\item El master realiza su trabajo asignado en dicha etapa, y al finalizar setea su registro RAX en cero y setea una variable bandera en 1 dando aviso de que finalizó y luego realiza polling al registro RAX mientras el valor sea cero.
    		\item El slave espera que el master finalice leyendo la variable bandera, esperando que su valor sea 1. Cuando esto ocurre, es enviada una interrupcion al master, y pasa a modo bajo consumo, es decir, se haltea.
    		\item El master recibe la interrupcion y en la ISR setea su registro RAX en 1, haciendo que salga del ciclo del item 2 para pasar a la siguiente etapa del algoritmo o finalizar segun corresponda.    		
    	\end{enumerate}


   	\begin{center}
        \includegraphics[height=6.5cm]{images/ipis-sorting-seq.png}
    \end{center}

        \newpage
         \subsection{Modificación de elementos de un arreglo}
	\subsubsection{Saturación del canal de memoria}
		Con el objetivo de analizar el impacto de los accesos a memoria de múltiples núcleos realizamos un experimento en el cual cada uno de los dos núcleos accede a una mitad del arreglo concurrentemente modificando los valores, realizando un incremento en uno del valor correspondiente.
		
	\begin{center}
 	    \includegraphics[height=4cm]{images/dualcore-vectorsum.pdf}
    \end{center}

	\subsubsection{Sincronización entre núcleos}
		En este algoritmo se realizo una sincronización sencilla de espera activa con roles master y slave, el protocolo es:
		\begin{enumerate}
			\item El master setea una bandera indicando al slave que comience su trabajo.
			\item El slave, que estaba realizando polling sobre la bandera del punto anterior comienza su trabajo, al finalizar, escribe en otra bandera indicando que finalizo.
			\item El master, luego de finalizar su trabajo, hace polling sobre una bandera en espera de la finalización del slave. Al salir de esta espera, concluye el algoritmo.
		\end{enumerate}
		
		\begin{center}
 	       \includegraphics[height=4cm]{images/sync-vectorsum-seq.png}
    	\end{center}

        \newpage
        \subsection{Fast Fourier Transform}
\subsubsection{Introducción al algoritmo}
El algoritmo de Cooley-Tukey, nombrado así por J.W. Cooley y John Tukey, es el algoritmo más común de transformada rápida de Fourier. El algoritmo expresa la transformada discreta de Fourier (DFT) de tamaño de un número compuesto N=N1N2 en términos de transformadas discretas de tamaños N1 y N2, de manera recursiva, para reducir su complejidad a O(N log N) para un número altamente compuesto N(smooth numbers). \\

\begin{center}
    \includegraphics[height=1cm]{images/fft_formula.png}
\end{center}

\subsubsection{Algoritmo de diezmado en el tiempo}

El algoritmo de diezmado en el tiempo de doble radix (decimation-in-time DIT) para la FFT es la forma más simple y común del algoritmo de Cooley–Tukey.
El algoritmo de Radix-2 DIT divide a la transformada discreta de Fourier de tamaño N en 2 transformadas discretas intercaladas (de ahí el nombre de doble radix o Radix-2) de tamaño N/2 en cada etapa recursiva.\\

Este algoritmo calcula inicialmente por separado los índices pares de los impares, nosotros aprovechamos esto para que los índices pares sean calculados por un core mientras los impares eran calculados por otro, de manera de aumentar la velocidad.\\

Este algoritmo es un caso interesante a estudiar debido a que constantemente los cores debían sincronizarse para trabajar, lo que nos permitió ver cómo se comportaban los diferentes métodos de sincronización con una alta demanda.\\

En la próxima sección de mostrarán resultados de la paralelización de este algoritmo.

\vspace{1cm}
Procesamiento de FFT:
\begin{center}
    \includegraphics[height=8cm]{images/Cooley-tukey-general.png}
\end{center}

\vspace{1cm}
Sincronización de FFT:
\begin{center}
    \includegraphics[height=8cm]{images/DIT-FFT-butterfly.png}
\end{center}

\subsubsection{Modificación en el cross-over para delegar trabajo a varios núcleos}
	Luego de las primeras pruebas, se modificó el parámetro que indica cuando comenzar a ejecutar el algoritmo de FFT en un solo núcleo a partir de cierto tamaño de entrada.
	Esto logró optimizar significativamente el rendimiento, mejorando además la brecha que había entre sincronización vía espera activa y sincronización vía interrupciones.
        \newpage
        \section{Resultados}
En esta sección se plasmarán los resultados obtenidos en las pruebas de las secciones anteriores, asimismo se darán detalles técnicos de los equipos donde fueron realizados los experimentos.
\subsection{Lectura e interpretación de resultados por pantalla}
El sistema luego de realizar la inicialización, comienza a arrojar por pantalla, los resultados(medidos en ticks) de los distintos algoritmos (separados por columna) aplicados sobre distintos tamaños de datos(separados por filas). Sobre estos datos obtenidos en una variedad de máquinas reales, se realizaron documentos, análisis y gráficos y serán plasmados debajo.
\subsection{Resultados: Forma de medición}
Las mediciones se realizaron utilizando los registros del procesador que contienen la cantidad de ticks transcurridos. Los puntos del código elegidos para comenzar y finalizar la medición son antes y después de llamar a los métodos de test, por ejemplo:
\begin{itemize}
	\item Comenzar medición
	\item Ejecutar sort\_single\_core\(\)
	\item Finalizar medición
\end{itemize}

Notemos que estas formas de medición no solo analizan el rendimiento del algoritmo propiamente dicho, sino además los tiempos de sincronización requeridos para cada implementación. (ver gráficos de los protocolos de sincronización), esto nos permitirá ver como en algunos casos, donde el tamaño de la entrada de datos es relativamente pequeña, la mejora obtenida en el multiprocesamiento es licuada por los tiempos extras requeridos para la sincronización, dándonos además un cross-over, el cual nos indica a partir de que tamaño de entrada comienza a ser mas ventajoso utilizar multiprocesamiento, permitiéndonos si lo deseáramos, realizar algoritmos mas inteligentes que dependiendo de los datos de entrada utilicen la implementación que tenga el mejor rendimiento.


\subsection{Resultados: Arquitectura de las máquinas utilizadas}
Se realizaron pruebas sobre los siguientes equipos:
\begin{enumerate}
	\item Intel® Pentium® Processor T4200
			\begin{itemize}
				\item CPU Clock: 2000Mhz
				\item Front Side Bus: 800Mhz
				\item Cores Number: 2
				\item L1 Caché: 2 x 32KB instruction, 2 x 32KB data
				\item L2 Caché: Shared 1MB
				\item Litografía: 45nm
			\end{itemize}
	\item Intel® Xeon® Processor E5345 
			\begin{itemize}
				\item CPU Clock: 2333Mhz
				\item Front Side Bus: 1333Mhz
				\item Cores Number: 4
				\item L1 Caché: 4 x 32KB instruction, 4 x 32KB data
				\item L2 Caché: 2 x 4MB shared
				\item Litografía: 65nm
			\end{itemize}
	\item Intel® Pentium® Processor G2030
			\begin{itemize}
				\item CPU Clock: 3000Mhz
				\item Direct Media Interface: 5 GT/s
				\item Cores Number: 2
				\item L1 Caché: 2 x 32KB instruction, 2 x 32KB data
				\item L2 Caché: 2 x 256K
				\item L3 Caché: 3 MB (Intel Smart Cache)
				\item Litografía: 22nm
			\end{itemize}
	\item Intel® Core™ i7-920 Processor 
			\begin{itemize}
				\item CPU Clock: 2666Mhz
				\item Bus Speed: 4.8 GT/s QPI (2400 MHz)
				\item Cores Number: 4
				\item L1 Caché: 4 x 32KB instruction, 4 x 32KB data
				\item L2 Caché: 4 x 256K
				\item L3 Caché: 8 MB (Intel Smart Cache) Shared
				\item Litografía: 45nm
			\end{itemize}
	\item Intel® Core™ i5-2500K Processor
			\begin{itemize}
				\item CPU Clock: 3300Mhz ~ 3700Mhz turbo
				\item Direct Media Interface: 5 GT/s
				\item Cores Number: 4
				\item L1 Caché: 4 x 32KB instruction, 4 x 32KB data
				\item L2 Caché: 4 x 256K
				\item L3 Caché: 6 MB (Intel Smart Cache) Shared
				\item Cache latency:
					\begin{itemize}	
						\item 4 (L1 cache)
						\item 11 (L2 cache)
						\item 25 (L3 cache)
					\end{itemize}
				\item Litografía: 32nm
			\end{itemize}
		\item Intel® Core™2 Quad Processor Q6600
			\begin{itemize}
				\item CPU Clock: 2400Mhz
				\item FSB: 1066Mhz
				\item Cores Number: 4
				\item L1 Caché: 4 x 32 KB 8-vias asociativa por conjuntos instrucciones, 4 x 32 KB 8-vias asociativa por conjuntos datos
				\item L2 Caché: 2 x 4 MB 16-vias asociativas (cada L2 cache es compartida por los 2 cores)				
				\item Litografía: 65nm
			\end{itemize}
\end{enumerate}
\textbf{Nota:} Todos los equipos tienen al menos 2 núcleos y su set de instrucciones es de 64 bits.
\subsection{Análisis de resultados}

\subsubsection{Intel® Pentium® Processor T4200 - Sorting}
\begin{center}
\begin{tabular}{|c|c|c|c|}
	\hline
		Elements & MonoCore Ticks & DualCore Ticks & MonoCore/DualCore Ratio\\
	\hline
		2 & 13880 & 48100 & 0.288\\
	\hline
		4 & 24270 & 49570 & 0.489\\
	\hline
		8 & 55400 & 91690 & 0.604\\
	\hline
		16 & 141500 & 168570 & 0.839\\
	\hline
		32 & 327920 & 353760 & 0.926\\
	\hline
		64 & 796010 & 742520 & 1.072\\
	\hline
		128 & 1849630 & 1679460 & 1.101\\
	\hline
		256 & 4320740 & 3646330 & 1.184\\
	\hline
		512 & 9934510 & 7628720 & 1.302\\
	\hline
		1024 & 22071820 & 16545320 & 1.334\\
	\hline
		2048 & 48763360 & 35257470 & 1.383\\
	\hline
		4096 & 106738920 & 76572280 & 1.393\\
	\hline
		8192 & 231258040 & 163781130 & 1.411\\
	\hline
		16384 & 498651230 & 347409350 & 1.435\\
	\hline
		32768 & 1068017890 & 735754710 & 1.451\\
	\hline
		65536 & 2273209280 & 1551108110 & 1.465\\
	\hline
		131072 & 4829745290 & 3277073490 & 1.473\\
	\hline
		262144 & 10221826090 & 6889769040 & 1.483\\
	\hline
		524288 & 21634980190 & 14443509820 & 1.497\\
	\hline
		1048576 & 45669905810 & 30263300000 & 1.509\\
	\hline
		2097152 & 96048564890 & 63479415710 & 1.513\\
	\hline
\end{tabular}
\end{center}
\begin{center}
    \includegraphics[height=6cm]{images/pentium_d_sorting.png}
\end{center}
En esta configuración podemos observar un cross-over en 64 elementos a partir de donde comienza a mejorar la performance utilizando dos cores. Para la mayor cantidad de elementos obtenemos una mejora aproximada del 50\%.
\subsubsection{Intel® Pentium® Processor T4200 - Vector modification}
\begin{center}
\begin{tabular}{|c|c|c|c|}
	\hline	
		Elements & MonoCore Ticks & DualCore Ticks & MonoCore Ticks/DualCore Ticks\\
	\hline	
		2 & 6528 & 103472 & 0.063\\
	\hline	
		4 & 8128 & 94672 & 0.085\\
	\hline	
		8 & 10656 & 106480 & 0.100\\
	\hline	
		16 & 14032 & 113328 & 0.123\\
	\hline	
		32 & 32688 & 77680 & 0.420\\
	\hline	
		64 & 60736 & 119088 & 0.510\\
	\hline	
		128 & 118432 & 187120 & 0.632\\
	\hline	
		256 & 237984 & 241904 & 0.983\\
	\hline	
		512 & 471664 & 382400 & 1.233\\
	\hline	
		1024 & 958848 & 801840 & 1.195\\
	\hline	
		2048 & 1862816 & 1618736 & 1.150\\
	\hline	
		4096 & 3616320 & 3241024 & 1.115\\
	\hline	
		8192 & 6855104 & 6129552 & 1.118\\
	\hline	
		16384 & 13412768 & 11785552 & 1.138\\
	\hline	
		32768 & 26840704 & 23272048 & 1.153\\
	\hline	
		65536 & 53739680 & 47434224 & 1.132\\
	\hline	
		131072 & 107571760 & 93427920 & 1.151\\
	\hline	
		262144 & 216317472 & 182467152 & 1.185\\
	\hline	
		524288 & 430617120 & 364908528 & 1.180\\
	\hline	
		1048576 & 860450928 & 745400768 & 1.154\\
	\hline	
		2097152 & 1730369088 & 1491881600 & 1.159\\
	\hline	
		4194304 & 3462956448 & 3000064512 & 1.154\\
	\hline
\end{tabular}
\end{center}
\begin{center}
    \includegraphics[height=6cm]{images/pentium_d_vectorsum.png}
\end{center}
Podemos observar como, incluso utilizando dos núcleos, la saturación del canal de memoria licua la performance mejorada del multiprocesamiento.

\subsubsection{Intel® Xeon® Processor E5345 - Sorting}
\begin{tabular}{|c|c|c|c|}
	\hline
		Elements & MonoCore Ticks & DualCore Ticks & MonoCore Ticks/DualCore Ticks\\
	\hline
		2 & 20657 & 34692 & 0.595\\
	\hline
		4 & 40565 & 47131 & 0.860\\
	\hline
		8 & 90559 & 73661 & 1.229\\
	\hline
		16 & 208635 & 137970 & 1.512\\
	\hline
		32 & 485933 & 287224 & 1.691\\
	\hline
		64 & 1096592 & 628369 & 1.745\\
	\hline
		128 & 2490712 & 1407854 & 1.769\\
	\hline
		256 & 5545589 & 3095624 & 1.791\\
	\hline
		512 & 12350142 & 6709311 & 1.840\\
	\hline
		1024 & 26958904 & 14815346 & 1.819\\
	\hline
		2048 & 58405928 & 32200161 & 1.813\\
	\hline
		4096 & 126362572 & 69358674 & 1.821\\
	\hline
		8192 & 271480069 & 149265949 & 1.818\\
	\hline
		16384 & 580168379 & 320822040 & 1.808\\
	\hline
		32768 & 1265098373 & 677863235 & 1.866\\
	\hline
		65536 & 2663953124 & 1447008150 & 1.841\\
	\hline
		131072 & 5597340728 & 3054159843 & 1.832\\
	\hline
		262144 & 11904206349 & 6435623397 & 1.849\\
	\hline
		524288 & 24800933899 & 13558975689 & 1.829\\
	\hline
		1048576 & 52289342196 & 28461202042 & 1.837\\
	\hline
		2097152 & 109516365196 & 59447473463 & 1.842\\
	\hline
\end{tabular}

	\begin{center}
	    \includegraphics[height=6cm]{images/XEON_Sorting.png}
	\end{center}

	Dadas las mejores prestaciones de caché y arquitectura de esta configuración, vemos que se obtiene un 84\% de mejora al realizar el multiprocesamiento al ordenar muestras de 2097152 elementos. El cross-over donde conviene utilizar multicore es muy bajo, desde 8 elementos ya es mas rápido el ordenamiento dual core. 

\subsubsection{Intel® Xeon® Processor E5345 - Vector modification}
\begin{center}
\begin{tabular}{|c|c|c|c|}
	\hline
		Elements & MonoCore Ticks & DualCore Ticks & MonoCore Ticks/DualCore Ticks\\
	\hline
		2 & 4900 & 16800 & 0.291\\
	\hline
		4 & 5831 & 17388 & 0.335\\
	\hline
		8 & 8211 & 17976 & 0.456\\
	\hline
		16 & 12985 & 19537 & 0.664\\
	\hline
		32 & 23198 & 26880 & 0.863\\
	\hline
		64 & 42511 & 35168 & 1.208\\
	\hline
		128 & 81543 & 55769 & 1.462\\
	\hline
		256 & 159663 & 94430 & 1.690\\
	\hline
		512 & 316463 & 210994 & 1.499\\
	\hline
		1024 & 630154 & 441224 & 1.428\\
	\hline
		2048 & 1256241 & 908292 & 1.383\\
	\hline
		4096 & 2509745 & 1836184 & 1.366\\
	\hline
		8192 & 5014814 & 3696385 & 1.356\\
	\hline
		16384 & 10030118 & 7422128 & 1.351\\
	\hline
		32768 & 20053355 & 14880299 & 1.347\\
	\hline
		65536 & 40102146 & 29786883 & 1.346\\
	\hline
		131072 & 80194310 & 59598644 & 1.345\\
	\hline
		262144 & 161790321 & 119193046 & 1.357\\
	\hline
		524288 & 327379871 & 238416367 & 1.373\\
	\hline
		1048576 & 656828004 & 478496431 & 1.372\\
	\hline
		2097152 & 1309034153 & 953788577 & 1.372\\
	\hline
		4194304 & 2619897161 & 1891070293 & 1.385\\
	\hline
\end{tabular}
\end{center}

	\begin{center}
	    \includegraphics[height=6cm]{images/xeon_vector_sum.png}
	\end{center}

	Nuevamente los accesos a memoria afectan la performance del multiprocesamiento.

\subsubsection{Intel® Pentium® Processor G2030 - Sorting}
\begin{center}
	\begin{tabular}{|c|c|c|c|}
		\hline	
			Elements & MonoCore Ticks & DualCore Ticks & MonoCore Ticks/DualCore Ticks\\
		\hline
			2 & 15180 & 22996 & 0.660\\
		\hline
			4 & 30704 & 33812 & 0.908\\
		\hline
			8 & 70096 & 54804 & 1.279\\
		\hline
			16 & 186620 & 104288 & 1.789\\
		\hline
			32 & 436216 & 215084 & 2.028\\
		\hline
			64 & 1011620 & 473460 & 2.136\\
		\hline
			128 & 2311748 & 1074916 & 2.150\\
		\hline
			256 & 5156888 & 2393136 & 2.154\\
		\hline
			512 & 11414660 & 5290992 & 2.157\\
		\hline
			1024 & 25167592 & 11469944 & 2.194\\
		\hline
			2048 & 54610712 & 25100072 & 2.175\\
		\hline
			4096 & 118042804 & 57037548 & 2.069\\
		\hline
			8192 & 252518012 & 123304440 & 2.047\\
		\hline
			16384 & 536777312 & 264044108 & 2.032\\
		\hline
			32768 & 1097052680 & 555262936 & 1.975\\
		\hline
			65536 & 2322030308 & 1185902828 & 1.958\\
		\hline
			131072 & 4932183084 & 2539728536 & 1.942\\
		\hline
			262144 & 10447157964 & 5135427448 & 2.034\\
		\hline
			524288 & 21599239404 & 10835600988 & 1.993\\
		\hline
			1048576 & 45271465128 & 22793636612 & 1.986\\
		\hline
			2097152 & 94724393340 & 47526712420 & 1.993\\
		\hline
			4194304 & 197880816576 & 99686391488 & 1.985\\
		\hline
	\end{tabular}
\end{center}
	\begin{center}
	    \includegraphics[height=6cm]{images/g2030_sorting.png}
	\end{center}

	Esta arquitectura nueva, nos sorprendió con un crossover de 8 elementos en el cual mejora la performance y a partir de 32 elementos aproximadamente se duplica el rendimiento utilizando multicore!
\subsubsection{Intel® Pentium® Processor G2030 - Vector modification}

\begin{center}
	\begin{tabular}{|c|c|c|c|}
		\hline	
			Elements & MonoCore Ticks & DualCore Ticks & MonoCore Ticks/DualCore Ticks\\
		\hline
			2 & 3844 & 7552 & 0.509\\
		\hline
			4 & 5260 & 7780 & 0.676\\
		\hline
			8 & 7996 & 10356 & 0.772\\
		\hline
			16 & 15036 & 12592 & 1.194\\
		\hline
			32 & 24732 & 18220 & 1.357\\
		\hline
			64 & 46612 & 32060 & 1.453\\
		\hline
			128 & 87408 & 55104 & 1.586\\
		\hline
			256 & 173268 & 128556 & 1.347\\
		\hline
			512 & 336340 & 198352 & 1.695\\
		\hline
			1024 & 682580 & 561132 & 1.216\\
		\hline
			2048 & 1356376 & 745136 & 1.820\\
		\hline
			4096 & 2959568 & 2184308 & 1.354\\
		\hline
			8192 & 6134040 & 4536436 & 1.352\\
		\hline
			16384 & 12473160 & 6453392 & 1.932\\
		\hline
			32768 & 24960936 & 18386228 & 1.357\\
		\hline
			65536 & 49965760 & 25624672 & 1.949\\
		\hline
			131072 & 99945548 & 74779612 & 1.336\\
		\hline
			262144 & 199857864 & 90879264 & 2.199\\
		\hline
			524288 & 399539592 & 285954068 & 1.397\\
		\hline
			1048576 & 799210112 & 381478298 & 2.095\\
		\hline
			2097152 & 1598536888 & 1196662984 & 1.335\\
		\hline
			4194304 & 3196868172 & 1601236716 & 1.996\\
		\hline
	\end{tabular}
\end{center}
	\begin{center}
	    \includegraphics[height=6cm]{images/g2030_vectorsum.png}
	\end{center}

	A pesar de como afectan los accesos a memoria, en algunos casos vemos que se rompen los esquemas y se obtienen performances del doble versus procesamiento mono-núcleo.

\subsubsection{Intel® Core™ i7-920 Processor - Sorting - Vector modification }
\begin{center}
	\begin{tabular}{|c|c|c|c|c|c|c|}
		\hline	
			Elements & Sort Mono & Sort Dual & Vector Mono & Vector Dual & Sort Ratio & Vector Mod Ratio\\
		\hline
			2 & 17668 & 26204 & 4912 & 9332 & 0.674 & 0.526\\
		\hline
			4 & 36888 & 38648 & 5692 & 10680 & 0.954 & 0.532\\
		\hline
			8 & 84260 & 62436 & 8524 & 11696 & 1.349 & 0.728\\
		\hline
			16 & 192596 & 116900 & 15160 & 16356 & 1.647 & 0.926\\
		\hline
			32 & 463176 & 250716 & 26856 & 20256 & 1.847 & 1.325\\
		\hline
			64 & 1062000 & 549240 & 50600 & 32240 & 1.933 & 1.569\\
		\hline
			128 & 2427412 & 1236440 & 97260 & 55740 & 1.963 & 1.744\\
		\hline
			256 & 5389400 & 2719116 & 192860 & 122600 & 1.982 & 1.573\\
		\hline
			512 & 11997744 & 5903640 & 381276 & 260856 & 2.032 & 1.461\\
		\hline
			1024 & 25525608 & 13065240 & 761760 & 545620 & 1.953 & 1.396\\
		\hline
			2048 & 55905976 & 28463616 & 1519960 & 1110780 & 1.964 & 1.368\\
		\hline
			4096 & 120254264 & 69237000 & 3031636 & 2260796 & 1.736 & 1.340\\
		\hline
			8192 & 259149012 & 132442380 & 6051280 & 4509000 & 1.956 & 1.342\\
		\hline
			16384 & 554582312 & 284005796 & 12106900 & 9029180 & 1.952 & 1.340\\
		\hline
			32768 & 1199133428 & 612601980 & 24212436 & 12154760 & 1.957 & 1.992\\
		\hline
			65536 & 2547819264 & 1297609520 & 48417380 & 24300516 & 1.963 & 1.992\\
		\hline
			131072 & 5385534272 & 2769704056 & 96870560 & 48611840 & 1.944 & 1.992\\
		\hline
			262144 & 11395843660 & 5840527220 & 193900280 & 97191900 & 1.951 & 1.995\\
		\hline
			524288 & 23990990684 & 12323090580 & 398251576 & 299541500 & 1.946 & 1.329\\
		\hline
			1048576 & 50408173440 & 26217339456 & 792920300 & 395143060 & 1.922 & 2.006\\
		\hline
			2097152 & 105680965696 & 54178541700 & 1578620280 & 793110860 & 1.950 & 1.990\\
		\hline
			4194304 & 221064067796 & 113389632780 & 3176314036 & 2371198400 & 1.949 & 1.339\\
		\hline
	\end{tabular}
\end{center}

	\begin{center}
	    \includegraphics[height=7cm]{images/i7-vectorsum-sorting.png}
	\end{center}

	Podemos observar aquí que en arquitecturas mas nuevas de intel, se va mejorando la performance promedio obtenida, cada vez mas cercana a duplicar la performance single core, a pesar de esto, los accesos a memoria siguen siendo un problema en algunos casos.

\subsubsection{Intel® Core™ i5-2500K Processor - Sorting}
\begin{center}
	\begin{tabular}{|c|c|c|c|}
		\hline	
			Elements & MonoCore Ticks & DualCore Ticks & MonoCore Ticks/DualCore Ticks\\
		\hline
			2 & 15384 & 24580 & 0.625\\
		\hline
			4 & 31660 & 33112 & 0.956\\
		\hline
			8 & 74324 & 53544 & 1.388\\
		\hline
			16 & 171640 & 100236 & 1.712\\
		\hline
			32 & 406144 & 216452 & 1.876\\
		\hline
			64 & 932468 & 458840 & 2.032\\
		\hline
			128 & 2110364 & 1100608 & 1.917\\
		\hline
			256 & 4475012 & 2502924 & 1.787\\
		\hline
			512 & 9904452 & 5447160 & 1.818\\
		\hline
			1024 & 21732200 & 12102956 & 1.795\\
		\hline
			2048 & 47346996 & 26551652 & 1.783\\
		\hline
			4096 & 102885844 & 58287424 & 1.765\\
		\hline
			8192 & 222250916 & 120969120 & 1.837\\
		\hline
			16384 & 476876272 & 258581740 & 1.844\\
		\hline
			32768 & 1019406688 & 548013260 & 1.860\\
		\hline
			65536 & 2173616224 & 1186136300 & 1.832\\
		\hline
			131072 & 4633428528 & 2471420552 & 1.874\\
		\hline
			262144 & 9056811460 & 5190092924 & 1.745\\
		\hline
			524288 & 20854472344 & 10876582315 & 1.917\\
		\hline
			1048576 & 43923320368 & 23475758752 & 1.871\\
		\hline
			2097152 & 94551848940 & 49553049284 & 1.908\\
		\hline
			4194304 & 198157210936 & 103593610452 & 1.912\\
		\hline
	\end{tabular}
\end{center}
	\begin{center}
	    \includegraphics[height=6cm]{images/i5_sorting.png}
	\end{center}

	Dado que este procesador es cercano al i7, podemos observar una performance similar respecto a los ratios de mejora en multiprocesamiento, tanto en los ordenamientos, como en la modicación de vectores en al siguiente sección.

\subsubsection{Intel® Core™ i5-2500K Processor - Vector modification}
\begin{center}
	\begin{tabular}{|c|c|c|c|}
		\hline	
			Elements & MonoCore Ticks & DualCore Ticks & MonoCore Ticks/DualCore Ticks\\
		\hline
			2 & 3544 & 6900 & 0.513\\
		\hline
			4 & 5288 & 7280 & 0.726\\
		\hline
			8 & 7808 & 9868 & 0.791\\
		\hline
			16 & 13296 & 12132 & 1.095\\
		\hline
			32 & 23920 & 16980 & 1.408\\
		\hline
			64 & 47044 & 32276 & 1.457\\
		\hline
			128 & 92228 & 52192 & 1.767\\
		\hline
			256 & 182716 & 105168 & 1.737\\
		\hline
			512 & 363804 & 205808 & 1.767\\
		\hline
			1024 & 716228 & 394484 & 1.815\\
		\hline
			2048 & 1433180 & 786560 & 1.822\\
		\hline
			4096 & 2870572 & 1513800 & 1.896\\
		\hline
			8192 & 5694568 & 2914168 & 1.954\\
		\hline
			16384 & 11378960 & 5836656 & 1.949\\
		\hline
			32768 & 22432852 & 16351152 & 1.371\\
		\hline
			65536 & 45314096 & 22864312 & 1.981\\
		\hline
			131072 & 90536832 & 65673712 & 1.378\\
		\hline
			262144 & 181141232 & 91528392 & 1.979\\
		\hline
			524288 & 362184776 & 262148544 & 1.381\\
		\hline
			1048576 & 724287244 & 530532932 & 1.365\\
		\hline
			2097152 & 1448749484 & 1061714556 & 1.364\\
		\hline
			4194304 & 2897714296 & 1462866292 & 1.980\\
		\hline
	\end{tabular}
\end{center}
	\begin{center}
	    \includegraphics[height=6cm]{images/i5_vector_sum.png}
	\end{center}

\subsubsection{Intel® Core™2 Quad Processor Q6600 Sorting}
\begin{center}
\begin{tabular}{|c|c|c|c|}
	\hline	
		Elements & MonoCore Ticks & DualCore Ticks & MonoCore Ticks/DualCore Ticks\\
	\hline
		2 & 6885 & 16218 & 0.424\\
	\hline
		4 & 13277 & 18530 & 0.716\\
	\hline
		8 & 32011 & 30183 & 1.060\\
	\hline
		16 & 76364 & 56958 & 1.340\\
	\hline
		32 & 181024 & 120487 & 1.502\\
	\hline
		64 & 415990 & 252671 & 1.646\\
	\hline
		128 & 949399 & 560303 & 1.694\\
	\hline
		256 & 2121048 & 1218075 & 1.741\\
	\hline
		512 & 4742838 & 2604884 & 1.820\\
	\hline
		1024 & 10433809 & 5599078 & 1.863\\
	\hline
		2048 & 22765235 & 12013475 & 1.894\\
	\hline
		4096 & 49466056 & 25624066 & 1.930\\
	\hline
		8192 & 107050318 & 54191427 & 1.975\\
	\hline
		16384 & 230695245 & 119820352 & 1.925\\
	\hline
		32768 & 492634993 & 242634115 & 2.030\\
	\hline
		65536 & 1054223219 & 535990263 & 1.966\\
	\hline
		131072 & 2249860461 & 1087838729 & 2.068\\
	\hline
		262144 & 4773926137 & 2274182829 & 2.099\\
	\hline
		524288 & 10087422167 & 4892424806 & 2.061\\
	\hline
		1048576 & 21270642148 & 10335385475 & 2.058\\
	\hline
		2097152 & 44667373988 & 21181628542 & 2.108\\
	\hline
		4194304 & 93526794057 & 44040620699 & 2.123\\
	\hline
\end{tabular}
\end{center}

	\begin{center}
	    \includegraphics[height=6cm]{images/core2quad_sorting.png}
	\end{center}

	Esta arquitectura, a pesar de ser mas antigua de los ix, nos muestra una buena performance utilizando multicore, de aproximadamente el doble, la mejora respecto de la cantidad de elementos en aumento es casi instantánea, a partir de 8 elementos.

\subsubsection{Intel® Core™2 Quad Processor Q6600 Vector modification}
\begin{center}
	\begin{tabular}{|c|c|c|c|}
		\hline	
			Elements & MonoCore Ticks & DualCore Ticks & MonoCore Ticks/DualCore Ticks\\
		\hline
				2 & 1717 & 5695 & 0.301\\
		\hline
				4 & 1938 & 6945 & 0.279\\
		\hline
				8 & 2456 & 6945 & 0.353\\
		\hline
				16 & 5491 & 9826 & 0.558\\
		\hline
				32 & 13745 & 13515 & 1.017\\
		\hline
				64 & 28016 & 18309 & 1.530\\
		\hline
				128 & 56151 & 32266 & 1.740\\
		\hline
				256 & 115379 & 68434 & 1.685\\
		\hline
				512 & 231107 & 150314 & 1.537\\
		\hline
				1024 & 459255 & 307930 & 1.491\\
		\hline
				2048 & 918060 & 623645 & 1.472\\
		\hline
				4096 & 1835176 & 1292867 & 1.419\\
		\hline
				8192 & 3682583 & 2606517 & 1.412\\
		\hline
				16384 & 7390283 & 5332679 & 1.385\\
		\hline
				32768 & 14743522 & 10669702 & 1.381\\
		\hline
				65536 & 29898164 & 20515192 & 1.457\\
		\hline
				131072 & 59790649 & 40836465 & 1.464\\
		\hline
				262144 & 119581808 & 83774122 & 1.427\\
		\hline
				524288 & 239160938 & 167648356 & 1.426\\
		\hline
				1048576 & 478114256 & 332301831 & 1.438\\
		\hline
				2097152 & 957119720 & 661399399 & 1.447\\
		\hline
				4194304 & 1913099785 & 1352226444 & 1.414\\
		\hline
	\end{tabular}
\end{center}
	\begin{center}
	    \includegraphics[height=6cm]{images/core2quad_sum.png}
	\end{center}

	Los accesos a memoria, a los cuales compiten los 2 cores, afectan el rendimiento, aunque utilizar dos cores sigue dando aproximadamente un 40\% de mejoría.
	
\subsubsection{Intel® Pentium® Processor T4200 - Sorting usando Inter-processor interrupts}

Para intentar reducir los accesos a memoria, implementamos la sincronización entre núcleos mediante interrupciones entre procesadores en lugar de realizar espera activa a memoria. Los resultados fueron curiosos, dado que dio peor la performance respecto a la sincronización con variables en memoria RAM. Creemos que esto se debe a las velocidades del bus de memoria y el bus de sistema y la memoria caché.

\begin{center}
	\begin{tabular}{|c|c|c|c|c|c|c|}
		\hline	
			Elements & Sort Single & Sort Dual Mem & Sort Dual Ipi & Ratio M/I & Single/DMem & Single/DIpi\\
		\hline
			2 & 26336 & 55088 & 123712 & 0.445  & 0.478  & 0.212 \\
		\hline
			4 & 51136 & 70896 & 164496 & 0.430  & 0.721  & 0.310 \\
		\hline
			8 & 114832 & 108224 & 272560 & 0.397  & 1.061 & 0.421 \\
		\hline
			16 & 268832 & 213984 & 363696 & 0.588  & 1.256 & 0.739 \\
		\hline
			32 & 613520 & 455376 & 571488 & 0.796  & 1.347 & 1.073 \\
		\hline
			64 & 1429152 & 951840 & 1063760 & 0.894  & 1.501 & 1.343 \\
		\hline
			128 & 2524576 & 2182176 & 2280608 & 0.956  & 1.156 & 1.106 \\
		\hline
			256 & 7816560 & 4642912 & 4782544 & 0.970  & 1.683 & 1.634 \\
		\hline
			512 & 17833216 & 9749392 & 10258304 & 0.950  & 1.829 & 1.738 \\
		\hline
			1024 & 37187824 & 20810160 & 22367904 & 0.930 & 1.787 & 1.662 \\
		\hline
			2048 & 79244560 & 44773584 & 49540224 & 0.903  & 1.769 & 1.599 \\
		\hline
			4096 & 171885520 & 95302656 & 107242560 & 0.888  & 1.803 & 1.602 \\
		\hline
			8192 & 369363168 & 191632704 & 221836560 & 0.863  & 1.927 & 1.665 \\
		\hline
			16384 & 781349776 & 401270912 & 473079168 & 0.848  & 1.947 & 1.651 \\
		\hline
			32768 & 1670361776 & 882252640 & 1065852080 & 0.827  & 1.893 & 1.567 \\
		\hline
			65536 & 3568050464 & 1873549072 & 2277199056 & 0.822  & 1.904 & 1.566 \\
		\hline
			131072 & 7538692864 & 3718695296 & 4572378768 & 0.813  & 2.027 & 1.648 \\
		\hline
			262144 & 15870953952 & 7744017984 & 9611314224 & 0.805  & 2.049 & 1.651 \\
		\hline
			524288 & 33266571168 & 16705027472 & 21105461872 & 0.791  & 1.991 & 1.576 \\
		\hline
			1048576 & 69926858336 & 34701409440 & 44269003616 & 0.783  & 2.015 & 1.579 \\
		\hline
			2097152 & 146290056320 & 72153753040 & 92620047520 & 0.779  & 2.027 & 1.579 \\
		\hline
			4194304 & 305621530000 & 149990136352 & 193768043584 & 0.774  & 2.037 & 1.577 \\
		\hline
	\end{tabular}
\end{center}

	\begin{center}
	    \includegraphics[height=7cm]{images/pentium_d_mem_ipi.png}
	\end{center}

	\begin{center}
	    \includegraphics[height=6cm]{images/pentium_d_gain_mono_mem_mono_ipi.png}
	\end{center}

\subsubsection{Intel® Pentium® Processor G2030 - Sorting usando Inter-processor interrupts}

\begin{center}
	\begin{tabular}{|c|c|c|c|c|c|c|}
		\hline	
			Elements & Sort Single & Sort Dual Mem & Sort Dual Ipi & Ratio M/I & Single/DMem & Single/DIpi\\
		\hline
			2 & 15452 & 21896 & 102700 & 0.213 & 0.705 & 0.150\\
		\hline
			4 & 29912 & 32196 & 110408 & 0.291 & 0.929 & 0.270\\
		\hline
			8 & 71836 & 50516 & 134456 & 0.375 & 1.422 & 0.534\\
		\hline
			16 & 169000 & 103400 & 183016 & 0.564 & 1.634 & 0.923\\
		\hline
			32 & 397952 & 211700 & 310164 & 0.682 & 1.879 & 1.283\\
		\hline
			64 & 916460 & 471424 & 565800 & 0.833 & 1.944 & 1.619\\
		\hline
			128 & 2139260 & 1090936 & 1181260 & 0.923 & 1.960 & 1.810\\
		\hline
			256 & 4788876 & 2416744 & 2549880 & 0.947 & 1.981 & 1.878\\
		\hline
			512 & 10815520 & 5308704 & 5487300 & 0.967 & 2.037 & 1.971\\
		\hline
			1024 & 23706116 & 11522124 & 11979292 & 0.961 & 2.057 & 1.978\\
		\hline
			2048 & 51881936 & 25373340 & 25050724 & 1.012 & 2.044 & 2.071\\
		\hline
			4096 & 112622828 & 56403280 & 57929704 & 0.973 & 1.996 & 1.944\\
		\hline
			8192 & 241852140 & 121972712 & 124943688 & 0.976 & 1.982 & 1.935\\
		\hline
			16384 & 515303872 & 262110928 & 266906044 & 0.982 & 1.965 & 1.930\\
		\hline
			32768 & 1138315716 & 556395036 & 564395308 & 0.985 & 2.045 & 2.016\\
		\hline
			65536 & 2328303276 & 1185656408 & 1198435060 & 0.989 & 1.963 & 1.942\\
		\hline
			131072 & 5102082740 & 2520384000 & 2552306004 & 0.987 & 2.024 & 1.999\\
		\hline
			262144 & 10786122552 & 5146255024 & 5192631012 & 0.991 & 2.095 & 2.077\\
		\hline
			524288 & 21983866092 & 10830150400 & 10945820376 & 0.989 & 2.029 & 2.008\\
		\hline
			1048576 & 46087236500 & 22837077648 & 23007268768 & 0.992 & 2.018 & 2.003\\
		\hline
			2097152 & 97382977980 & 47855259656 & 48355560760 & 0.989 & 2.034 & 2.013\\
		\hline
			4194304 & 203101063672 & 100389365112 & 100937254332 & 0.994 & 2.023 & 2.012\\
		\hline
	\end{tabular}
\end{center}

\begin{center}
	    \includegraphics[height=7cm]{images/memory_ipis_g2030_sortipi.png}
	\end{center}


\begin{center}
	    \includegraphics[height=6cm]{images/sort_ipis_g2030.png}
	\end{center}


Dada la velocidad superior del bus de sistema y las nuevas tecnologías de caché de la arquitectura se ve una mejoría respecto al procesador Pentium D T4200 al momento de comparar sincronización entre memoria e ipis. 

\subsubsection{Intel® Pentium® Processor T4200 - FFT Single - FFT Dual Mem - FFT Dual IPI}

\begin{center}
	\begin{tabular}{|c|c|c|c|c|c|}
		\hline	
			Elements & FFT Single & FFT Dual Mem & FFT Dual Ipi & Single/DMem & Single/DIpi\\
		\hline
			2 & 70832 & 92672 & 94400 & 0.764 & 0.750\\
		\hline
			4 & 178464 & 229536 & 284256 & 0.777 & 0.627\\
		\hline
			8 & 420560 & 517824 & 671888 & 0.812 & 0.625\\
		\hline
			16 & 1247232 & 1132416 & 1502656 & 1.101 & 0.830\\
		\hline
			32 & 2860176 & 2475856 & 3286000 & 1.155 & 0.870\\
		\hline
			64 & 6449744 & 5451264 & 7063456 & 1.183 & 0.913\\
		\hline
			128 & 14460480 & 11873856 & 15290304 & 1.217 & 0.945\\
		\hline
			256 & 32076304 & 25889856 & 32869536 & 1.238 & 0.975\\
		\hline
			512 & 70559488 & 56002208 & 70211568 & 1.259 & 1.004\\
		\hline
			1024 & 154010400 & 120728848 & 149749936 & 1.275 & 1.028\\
		\hline
			2048 & 323222368 & 257452224 & 317900272 & 1.255 & 1.016\\
		\hline
			4096 & 701196608 & 552926256 & 675095392 & 1.268 & 1.038\\
		\hline
			8192 & 1521703040 & 1185646304 & 1429235616 & 1.283 & 1.064\\
		\hline
			16384 & 3270511200 & 2522381008 & 3018106480 & 1.296 & 1.083\\
		\hline
	\end{tabular}
\end{center}


\begin{center}
	    \includegraphics[height=6cm]{images/fft_pentiumd.png}
	\end{center}

	Se obtiene una mejora utilizando dual core a partir de los 16 elementos, culminando en un 30\% de mejora en la máxima cantidad testeada. En esta arquitectura el uso de IPIS no es muy favorable respecto a la sincronización por memoria.


\subsubsection{Intel® Core™2 Quad Processor Q6600 - FFT Single - FFT Dual Mem - FFT Dual IPI}

\begin{center}
	\begin{tabular}{|c|c|c|c|c|c|}
		\hline	
			Elements & FFT Single & FFT Dual Mem & FFT Dual Ipi & Single/DMem & Single/DIpi\\
		\hline
			2 & 28815 & 29427 & 29385 & 0.979 & 0.980\\
		\hline
			4 & 72845 & 70329 & 101031 & 1.035 & 0.721\\
		\hline
			8 & 173103 & 155848 & 194199 & 1.110 & 0.891\\
		\hline
			16 & 404991 & 344539 & 433483 & 1.175 & 0.934\\
		\hline
			32 & 924281 & 750856 & 949297 & 1.230 & 0.973\\
		\hline
			64 & 2098420 & 1639624 & 2054917 & 1.279 & 1.021\\
		\hline
			128 & 4705387 & 3568818 & 4415665 & 1.318 & 1.065\\
		\hline
			256 & 10448327 & 7752374 & 9476531 & 1.347 & 1.102\\
		\hline
			512 & 23017286 & 16735140 & 20251377 & 1.375 & 1.136\\
		\hline
			1024 & 50295528 & 36675672 & 43080728 & 1.371 & 1.167\\
		\hline
			2048 & 109147616 & 77031496 & 91376640 & 1.416 & 1.194\\
		\hline
			4096 & 235491488 & 165446541 & 193718867 & 1.423 & 1.215\\
		\hline
			8192 & 505322288 & 350143806 & 408601570 & 1.443 & 1.236\\
		\hline
			16384 & 1080310024 & 740571382 & 860584064 & 1.458 & 1.255\\
		\hline
	\end{tabular}
\end{center}

\begin{center}
	    \includegraphics[height=6cm]{images/fft_core2quad.png}
	\end{center}

Vemos como en esta arquitectura mejora contra la anterior la mejora máxima, de un 46\%. Las ipis todavía no son demasiado favorables, además se observa que el cross-over para la mejora en dual core es casi instantáneo sobre la cantidad de elementos.

\subsubsection{Intel® Core™ i5-2500K Processor - FFT Single - FFT Dual Mem - FFT Dual IPI}

\begin{center}
	\begin{tabular}{|c|c|c|c|c|c|}
		\hline	
			Elements & FFT Single & FFT Dual Mem & FFT Dual Ipi & Single/DMem & Single/DIpi\\
		\hline
			2 & 77784 & 82284 & 78968 & 0.945 & 0.985\\
		\hline
			4 & 214688 & 191236 & 216976 & 1.122 & 0.989\\
		\hline
			8 & 514696 & 428148 & 502068 & 1.202 & 1.025\\
		\hline
			16 & 1215252 & 927192 & 1114204 & 1.310 & 1.090\\
		\hline
			32 & 2802900 & 2014076 & 2419752 & 1.391 & 1.158\\
		\hline
			64 & 6360344 & 4367528 & 5236460 & 1.456 & 1.214\\
		\hline
			128 & 14268528 & 9535292 & 11246792 & 1.496 & 1.268\\
		\hline
			256 & 31695828 & 20575104 & 24102060 & 1.540 & 1.315\\
		\hline
			512 & 69969592 & 44330152 & 51512984 & 1.578 & 1.358\\
		\hline
			1024 & 153126636 & 95175168 & 109542068 & 1.608 & 1.397\\
		\hline
			2048 & 332522532 & 203051852 & 231857552 & 1.637 & 1.434\\
		\hline
			4096 & 717823008 & 432009528 & 489417964 & 1.661 & 1.466\\
		\hline
			8192 & 1541572072 & 916023644 & 1030637576 & 1.682 & 1.495\\
		\hline
			16384 & 3296190392 & 1937514232 & 2166612988 & 1.701 & 1.521\\
		\hline
	\end{tabular}
\end{center}

\begin{center}
	    \includegraphics[height=6cm]{images/fft_i5.png}
	\end{center}
En la arquitectura del core i5, mejora el rendimiento obtenido utilizando multicore, otorgando un 70\% de mejora versus monoprocesamiento. Las ipis proveen un 52\% de mejora, además observamos que el cross-over para la mejora en dual core es casi instantáneo sobre la cantidad de elementos.

\subsubsection{Intel® Pentium® Processor G2030 - FFT Single - FFT Dual Mem - FFT Dual IPI}

\begin{center}
	\begin{tabular}{|c|c|c|c|c|c|}
		\hline	
			Elements & FFT Single & FFT Dual Mem & FFT Dual Ipi & Single/DMem & Single/DIpi\\
		\hline
			2 & 75888 & 80812 & 77972 & 0.939 & 0.973\\
		\hline
			4 & 209524 & 191800 & 220424 & 1.092 & 0.950\\
		\hline
			8 & 514828 & 426920 & 510424 & 1.205 & 1.008\\
		\hline
			16 & 1205000 & 938892 & 1142776 & 1.283 & 1.054\\
		\hline
			32 & 2760068 & 2047208 & 2490212 & 1.348 & 1.108\\
		\hline
			64 & 6291332 & 4508696 & 5387504 & 1.395 & 1.167\\
		\hline
			128 & 14026052 & 9758488 & 11629396 & 1.437 & 1.206\\
		\hline
			256 & 31301112 & 21347084 & 24994892 & 1.466 & 1.252\\
		\hline
			512 & 69070356 & 46043584 & 53512760 & 1.500 & 1.290\\
		\hline
			1024 & 150448296 & 98699668 & 113700712 & 1.524 & 1.323\\
		\hline
			2048 & 325692216 & 211210216 & 241297652 & 1.542 & 1.349\\
		\hline
			4096 & 702220892 & 450242892 & 511164280 & 1.559 & 1.373\\
		\hline
			8192 & 1507872220 & 957204752 & 1079792760 & 1.575 & 1.396\\
		\hline
			16384 & 3222351048 & 2028073172 & 2273867600 & 1.588 & 1.417\\
		\hline
	\end{tabular}
\end{center}


\begin{center}
	    \includegraphics[height=6cm]{images/fft_g2030.png}
	\end{center}

Aquí obtenemos una mejora máxima de un 59\%, el cross-over es casi instantáneo y mejoran bastante las ipis, en relación a su contra parte sincronizada por espera activa con accesos a memoria al compararse con monoprocesamiento.

\subsubsection{Intel® Xeon® Processor E5345 - FFT Single - FFT Dual Mem - FFT Dual IPI}

\begin{center}
	\begin{tabular}{|c|c|c|c|c|c|}
		\hline	
			Elements & FFT Single & FFT Dual Mem & FFT Dual Ipi & Single/DMem & Single/DIpi\\
		\hline
			2 & 93429 & 97160 & 97314 & 0.961 & 0.960\\
		\hline
			4 & 245973 & 231707 & 271544 & 1.061 & 0.905\\
		\hline
			8 & 597730 & 509642 & 627865 & 1.172 & 0.952\\
		\hline
			16 & 1394330 & 1109801 & 1382430 & 1.256 & 1.008\\
		\hline
			32 & 3210844 & 2413852 & 2992248 & 1.330 & 1.073\\
		\hline
			64 & 7297927 & 5258589 & 6447819 & 1.387 & 1.131\\
		\hline
			128 & 16403002 & 11400445 & 13845454 & 1.438 & 1.184\\
		\hline
			256 & 36485239 & 24669883 & 29654149 & 1.478 & 1.230\\
		\hline
			512 & 80414341 & 53201029 & 63250026 & 1.511 & 1.271\\
		\hline
			1024 & 175866670 & 114169888 & 134506631 & 1.540 & 1.307\\
		\hline
			2048 & 383665926 & 244086430 & 284939466 & 1.571 & 1.346\\
		\hline
			4096 & 831156431 & 512455125 & 590454613 & 1.621 & 1.407\\
		\hline
			8192 & 1799559559 & 1083358444 & 1262194136 & 1.661 & 1.425\\
		\hline
			16384 & 3789572983 & 2328605440 & 2658443333 & 1.627 & 1.425\\
		\hline
	\end{tabular}
\end{center}

\begin{center}
	    \includegraphics[height=5cm]{images/fft_xeon_e5345.png}
\end{center}

En esta máquina se obtuvo un valor máximo de un 63\% de mejora entre single y dual core con espera activa, el cross-over es a partir de 4 elementos y la arquitectura brinda una mejora de 42\% respecto a las ipis contra single core.

\subsubsection{Intel® Pentium® Processor G2030 - FFT Single - FFT Dual Mem - FFT Dual IPI - Corregido parámetro cross-over}

\begin{center}
	\begin{tabular}{|c|c|c|c|c|c|}
		\hline	
			Elements & FFT Single & FFT Dual Mem & FFT Dual Ipi & Single/DMem & Single/DIpi\\
		\hline
			2 & 76928 & 80184 & 77008 & 0.959 & 0.998\\
		\hline
			4 & 209344 & 209876 & 209620 & 0.997 & 0.998\\
		\hline
			8 & 509544 & 504388 & 508344 & 1.010 & 1.002\\
		\hline
			16 & 1204652 & 1095428 & 1136104 & 1.099 & 1.060\\
		\hline
			32 & 2769940 & 2375436 & 2468452 & 1.166 & 1.122\\
		\hline
			64 & 6285460 & 5116232 & 5351456 & 1.228 & 1.174\\
		\hline
			128 & 14118780 & 11058380 & 11609068 & 1.276 & 1.216\\
		\hline
			256 & 31421656 & 23805852 & 24912928 & 1.319 & 1.261\\
		\hline
			512 & 69159148 & 50957916 & 53272104 & 1.357 & 1.298\\
		\hline
			1024 & 150365428 & 108450908 & 113057384 & 1.386 & 1.329\\
		\hline
			2048 & 325566300 & 230628592 & 239963220 & 1.411 & 1.356\\
		\hline
			4096 & 701862512 & 488929320 & 508322008 & 1.435 & 1.380\\
		\hline
			8192 & 1507207556 & 1034818916 & 1075238928 & 1.456 & 1.401\\
		\hline
			16384 & 3220958472 & 2182392616 & 2262842272 & 1.475 & 1.423\\
		\hline
	\end{tabular}
\end{center}

\begin{center}
	    \includegraphics[height=6cm]{images/fft_g2030_corrected.png}
\end{center}

\subsubsection{Intel® Pentium® Processor T4200 - FFT Single - FFT Dual Mem - FFT Dual IPI - Corregido parámetro cross-over}

\begin{center}
	\begin{tabular}{|c|c|c|c|c|c|}
		\hline	
			Elements & FFT Single & FFT Dual Mem & FFT Dual Ipi & Single/DMem & Single/DIpi\\
		\hline
			2 & 35761 & 35805 & 35486 & 0.998 & 1.007\\
		\hline
			4 & 90915 & 91388 & 92862 & 0.994 & 0.979\\
		\hline
			8 & 217316 & 214940 & 219373 & 1.011 & 0.990\\
		\hline
			16 & 509124 & 465883 & 501732 & 1.092 & 1.014\\
		\hline
			32 & 1173953 & 1007820 & 1092289 & 1.164 & 1.074\\
		\hline
			64 & 2669315 & 2171323 & 2385240 & 1.229 & 1.119\\
		\hline
			128 & 5989203 & 4716437 & 5155986 & 1.269 & 1.161\\
		\hline
			256 & 13300727 & 10159050 & 11098527 & 1.309 & 1.198\\
		\hline
			512 & 29302328 & 21799811 & 23799919 & 1.344 & 1.231\\
		\hline
			1024 & 64078740 & 46520694 & 50830560 & 1.377 & 1.260\\
		\hline
			2048 & 139116241 & 99224873 & 108137887 & 1.402 & 1.286\\
		\hline
			4096 & 302052850 & 212169727 & 230777096 & 1.423 & 1.308\\
		\hline
			8192 & 648160260 & 447351124 & 486158453 & 1.448 & 1.333\\
		\hline
			16384 & 1385888295 & 944842954 & 1026775189 & 1.466 & 1.349\\
		\hline
	\end{tabular}
\end{center}

\begin{center}
    \includegraphics[height=6cm]{images/fft_pentiumd_corrected.png}
\end{center}

        \newpage
    %%%%%%%%%%%%%%%%%%%%%%%%%%%%%%%%%%%%%%%%%%%%%%%%%%%%%%%%%%%%%%%%%%%%%%%%%%%%%%%
    %% Conclusion                                                                %%
    %%%%%%%%%%%%%%%%%%%%%%%%%%%%%%%%%%%%%%%%%%%%%%%%%%%%%%%%%%%%%%%%%%%%%%%%%%%%%%%
    \section{Conclusión Final}
        Como conclusión de este trabajo, pensamos que la performance obtenida al utilizar dual core a un nivel muy bajo para algoritmos paralelizables no es para nada despreciable, dado que obtuvimos mejoras entre 50\% y 100\% dependiendo de la arquitectura del hardware donde se realizaron los experimentos, esto es una mejor performance que utilizando la infraestructura de threads de lenguajes de alto nivel. Restaría investigar mas aplicaciones en las cuales se pueda aprovechar este modelo ultra eficiente de cómputo y realizar pruebas con sistemas operativos reales que nos permitan verificar que el overhead creado por estos sea suficientemente apreciable como para decidir utilizar nuestra herramienta.

\end{document}